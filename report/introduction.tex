max 2 pages:
\begin{itemize}
\item Description of problem area and problem itself
\item What is the research question/goal?
\item Why is this is an important/meaningful/interesting problem to consider?
\item the very basic idea of the approach and why is this a reasonable approach for this problem?
\end{itemize}

\subsection{Description of problem area}
Music genre classification has been researched mostly using audio signals. While this approach uses the \textit{defining} characteristic of music, namely, its musical structure, an other interesting approach is to use word-based analysis.

\subsection{Research question}
Can an extension of LDA over music genres instead of documents be used to classify genres?

\subsection{Why is this important etc}
Online music services consists of large databases of meta-data for songs and artists. However, most of the information on such websites is crowd sourced and thus error-prone, but needed to suggest artists to users based on their earlier interests. Since obscure artists and releases are often unlabeled, and the problem with manual labeling is that it is time and cost expensive, automated classification of music genre would provide music services with a cost efficient way of labeling.