\subsection{Challenges}
The main challenge in this research was caused by the dataset: more than half of the dataset is labeled as \textit{Pop/Rock}, a label that unites both easy-to-listen top-40 songs and brutal death metal songs. The diversity of this genre makes it quite hard to classify, since such a broad genre contains a lot of different words and therefore a pretty even distribution over topics. 

Another challenge was the fact that some words, such as \textit{love}, occur often in a lot of different genres. Words that are common in many different genres make topic distributions more even, and are thus low-information words. These should be filtered from the dataset along with other low-information words (such as `girl' and `baby').

\subsection{Conclusion}
Concluding, the extended version of LDA as proposed in this paper forms sensible topics for different genres, especially those genres named after their lyrical themes. However, due to the very uneven distribution of documents over genres (some genres dominated the dataset by sheer number), classification did not work as well as expected. Extended LDA did outperform regular LDA, and is thus a better choice for genre classification tasks, which could also be applied to, for example, literary genres.

The question researched in this paper cannot be answered conclusively: preliminary results show that the classifier does not perform very well, even for the better version of LDA. However, it is assumed that this underperformance is largely due to a poorly labeled dataset. 


\subsection{Future work}
If reseach is continued with the current dataset, the \textit{Pop/Rock} genre has to be split into more (sub)genres. This could be achieved by taking the topic distribution of subgenres, and merging subgenres containing similar topic distributions (for example - \textit{Heavy Metal} and \textit{Speed Metal} are usually not far apart thematically). This way, subgenres like \textit{Death Metal} and \textit{Punk Rock} should be separated from, for example, top 40 songs. 

Furthermore, words that occur a lot in different genres should be removed from the dataset. A proposed method for achieving this is by storing the number of occurrences of a word for each genre, and if the word occurs in more than a threshold percentage of the genres, remove it from the dataset.

Another proposed addition to this research is to perform a 10-fold cross-validation (instead of 5-fold cross-validation) and run it over multiple experiments. Due to time constraints, this extensive measuring was not within the scope of this research, but it can strengthen the conclusions drawn from the evaluations of the classifications. \\
More tests have to be run, especially tests where more topics are extracted. Since the experiment with $\alpha = 0.5$ and $\beta = 0.1$ provided the best results for different parameter settings, and the experiment with 100 topics outperformed experiments with less topics, it is expected that a run with 100 topics instead of 50 with $\alpha$ set to 0.5 and $\beta$ set to 0.1 will yield even better results.