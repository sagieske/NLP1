\subsection{Challenges}
One of the main challenges of the research was caused by the dataset: the dataset consists for a large part of \verb|pop/rock| songs, in which both easy-to-listen top 40 songs are included, but also death metal songs. The diversity of this genre makes it quite hard to classify, since such a broad genre contains a lot of different words and therefore a pretty even distribution over topics. \\
Another challenge faced was the fact that some words, like \textit{love} occur in a lot of different genres. Words that occur in a lot of different genres make it harder to have a topic distribution with a high probabilty for only a couple of topics and should therefore be filtered from the dataset. 

\subsection{Conclusion}
It can be concluded that the extended version of LDA forms sensible topics for different genres, especially the genres named after their content. However, due to the very uneven distribution of lyrics over genres, it became hard to classify the test set and reach a high F1 score. The extended version of LDA did perform better than regular LDA, which is a sign that extended LDA with Gibbs sampling is a sensible choice to form topics based upon genres. \\
Using a SVM with the topic distributions over genres, an average correct prediction was given in about 57\% of the lyrics in the test set. Using a better distributed data set with more distinct genres, this number should increase. Still, the performance of the algorithm was way better than the performance of a baseline classifier, which predicted no more than 42\% correct lyric genres.

\subsection{Future work}
To continue reseach with the current dataset, the \verb|pop/rock| genre has to be split into more (sub)genres. This could be achieved by taking the topic distribution of subgenres, and merging subgenres containing similair topic distributions. This way, subgenres like black metal and death metal should be splitted from, for example, top 40 songs. \\
Furthermore, words that occur a lot in different genres should be removed from the dataset. This can be done by keeping the count of occurences of a word in different genres, and if the word occurs in more than a certain percentage of the genres, remove it from the dataset.
Another addition for this research is to perform a 10-fold cross-validation (instead of 5-fold cross-validation) and run it for multiple experiments. Due to time constraints, this extensive measuring was not in the scope of our project but it can strengthen evaluations of the classifications. \\
More tests have to be run, especially with more topics. The experiment with $\alpha = 0.5$ and $\beta = 0.1$ provided the best results, it is expected that a run with 100 topics instead of 50 will yield even better results.