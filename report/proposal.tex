\documentclass[12pt,a4paper]{amsart}
\usepackage[latin1]{inputenc}
\usepackage{graphicx}
\usepackage{amsmath}
\usepackage{amsfonts}
\usepackage{amssymb}
\usepackage{url}
\usepackage{cite}
\usepackage{natbib}
\usepackage{colortbl}
\usepackage[usenames,dvipsnames]{xcolor}
\title{Proposal NLP Mini-project}
\author{David van Erkelens, Sharon Gieske, Elise Koster}

\usepackage{fancyhdr}
\usepackage{booktabs}


% Set par indents to 0
\setlength{\parindent}{0cm}

% Set header for every page (except for first page)
\pagestyle{fancy}
\rhead[\small\textsc{David van Erkelens, Sharon Gieske, Elise Koster}]{\small\textsc{Proposal NLP Mini-project}}
\lhead{\thepage.}
\cfoot{}
\date{}

\begin{document}
\maketitle

\section{Introduction}
This proposal describes the outline of a project researching the application of LDA and rhyme schemes to music (sub-) genre classification.\\
Nowadays, many online music services exist, where users want music classified by genre. However, if such a service misses this information, they could easily retrieve an estimated genre using lyrics-based genre classification. 
Most musical genre classification research was based on audio signals, which is more computationally expensive, requires more storage and is less noise-sensitive. \\


\section{Objectives}
The goal of this project is modeling the topic distributions for multiple music genres using lyrics. 
Research questions:
\begin{itemize}
\item What are the most unique words and topics per musical genre?
\item Are topic models a suitable feature for music genre classification using SVMs?
\item Are topic models a suitable feature for subgenre classification using SVMs?
\item Are rhyme schemes discriminative features of music genres?
\item Which music genres correlate most and least?
\end{itemize}

\section{Approach}
To answer the research questions, the following steps will be used: first, a data set will be collected (possibly using a crawler, and online music websites) and processed (to clean it of noise and common words like `the'). Then, LDA (topic model) will be implemented and used to create a distribution of topics per music genre. These distributions will then be used as a feature in training a multi-class SVM. If there is time after these results are obtained, a rhyme scheme library will be added and the rhyme schemes will be used as a feature in training another multi-class SVM. These results will be compared to the topic model classification and evaluated in a report. Also, a presentation will be prepared to concisely summarize the project and its results.

\section{Deliverables}
The components delivered upon completing this project will be a program that classifies a lyrics-document into a (sub-) genre, a report documenting the results of the project, the answers to the research questions and outlining possible future areas of research and a presentation reporting the work of the project, possibly featuring a live demonstration of the program.
\section{Planning}

\definecolor{dark-gray}{gray}{0.25}
\begin{tabular}{ |c | c |}
\rowcolor{dark-gray}
\hline
\textcolor{white}{\textbf{Subject}} & \textcolor{white}{\textbf{Week}}\\
	\hline
	Gathering dataset and literature & week 1  \\
	\hline
	Clean and visualize dataset & week 2 \\
	\hline
	Implement LDA & week 3 \\
	\hline
	Implement LDA with SVM & week 4 \\
	\hline
	Add rhyme scheme feature & week 5 \\
	\hline
	Start report and presentation & week 6 \\
	\hline
	Finish report and presentation & week 7\\
	\hline
\end{tabular}\\

\bibliography{../report/bibliography}
\bibliographystyle{plain}

\end{document}