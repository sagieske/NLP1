%roughly 1-2 pages
%\begin{itemize}
%\item explain the problem; what kind of assumptions/observations you have about the problem
%\end{itemize}

Automated genre classification based on lyrics has been attempted before \cite{fell lyrics}, using a large range of features\footnote{$n$-grams, vocabulary, syntactic structure, semantics and more}. However, none of these features used topic modeling or an extension thereof. Moreover, topic modeling \textit{has} been used for music analysis, but not in the context of genre classification \cite{lukiccomparison}. Thus, no research has been done using topics extracted using LDA (or an extension thereof) for classification. \\
Topic modeling is attempted partly because of the observations that different genres deal with different kinds of subjects. So much so, that stereotypes are based around fans of different music genres. For example, within the scope of the \textit{heavy metal} genre, \textit{death metal} is a genre that often focuses around death, gore, rot and murder, from either the victim or the perpetrator's point of view, while, \textit{black metal} usually alludes to satanism, anti-Christianity and misanthropy. On the other hand, a genre like \textit{rap} is stereotyped to be focussed on sex, drugs, gangs and violence, whereas \textit{reggae} is usually regarded to be about Rastafari culture and smoking marijuana. Moreover, certain music genres are named after the content of their lyrics; the \textit{holiday} genre primarily deals with Christmas, whereas \textit{religious} contains all kinds of music, as long as it deals with religious themes. An important side note is that certain music genres, such as \textit{techno}, \textit{trance} and \textit{house} (with the exception of their \textit{vocal} subgenres) usually don't contain any lyrics. As such, the approach used in this paper is unsuited for classification of these types of music. \\
The approach used in this paper assumes \textit{a)} that words are significant markers of a music genre, \textit{b)} that there is a significant difference between words in different music genres and \textit{c)} that topic distributions are accurate representations of lyrical content within a genre. \\
Since LDA is a generative model, it can also be used to create new lyrics. To create a `song' of length $n$ for a genre $G$, for each word position from $0$ to $n$, a topic $k$ is sampled from $G$'s topic distribution. Then, a word is sampled from $k$'s word distribution (see also algorithm \ref{alg:create-song}). \\
\begin{mdframed}
\begin{algorithm}[H]\label{alg:create-song}
 \KwData{Topic Model inferred by extended DA, genre $G$, length $n$}
 \KwResult{Song for genre $G$ of length $n$}
 initialize: song = `' \\
 \While{counter $< n$ }{
  sample $k \sim \theta_G$\\
  sample $w \sim \varphi_k$\\
  append $w$ to song\\
  counter ++ \
 }
 return song
 \caption{Song generation}
\end{algorithm}
\end{mdframed}