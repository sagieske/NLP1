%roughly 1-2 pages
%\begin{itemize}
%\item explain the problem; what kind of assumptions/observations you have about the problem
%\end{itemize}

Automated genre classification based on lyrics has been attempted before \cite{fell lyrics}, using a large range of features\footnote{$n$-grams, vocabulary, syntactic structure, semantics and more}. However, none of these features used topic modeling or an extension thereof. Moreover, topic modeling \textit{has} been used for music analysis, but not in the context of genre classification \cite{lukiccomparison} \cite{liartists}. Thus, no research has been done using topics extracted using LDA (or an extension thereof) for genre classification. 

Topic modeling is attempted partly because of the observations that different genres deal with different kinds of subjects. So much so, that stereotypes exist of fans of different music genres. For example, within the scope of the \textit{Heavy Metal} genre, \textit{Death Metal} is a genre that often focuses around death, gore, rot and murder, from either the victim or the perpetrator's point of view, while, \textit{Black Metal} usually alludes to satanism, anti-Christianity and misanthropy. On the other hand, a genre like \textit{Rap} has the stereotype of revolving around sex, drugs, gangs and violence, whereas \textit{Reggae} is usually regarded to be about Rastafari culture and smoking marijuana. Moreover, certain music genres are named after the content of their lyrics; the \textit{Holiday} genre primarily deals with Christmas, whereas the \textit{Religious} genre contains all kinds of music, as long as the music deals with religious themes. An important side note is that certain music genres, such as \textit{Techno}, \textit{Trance} and \textit{House} (with the exception of their \textit{Vocal} subgenres) usually don't contain any lyrics. As such, the approach used in this paper is unsuited for classification of these types of music. 

On the other hand, topic modeling is a \textit{generative} approach, meaning that the models built can be used to create new `songs' fitting of a genre. However, since the model used in this paper is a bag-of-words model, the resulting songs will be unstructured. Also, since low-information, common words, are assumed to not contribute much to topics, they are left out, and thus the generated songs will lack these.\\
The approach used in this paper assumes \textit{a)} that words are significant markers of a music genre, \textit{b)} that there is a significant difference between words in different music genres, \textit{c)} that topic distributions are accurate representations of lyrical content within a genre and \textit{d)} that common words\footnote{for example, `I', 'was', 'to'} are low-information and can thus be left out of the dataset. 
